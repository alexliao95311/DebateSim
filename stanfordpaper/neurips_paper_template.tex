\documentclass[11pt]{article}

% NeurIPS 2024 template
\usepackage[utf8]{inputenc}
\usepackage[T1]{fontenc}
\usepackage{hyperref}
\usepackage{url}
\usepackage{booktabs}
\usepackage{amsfonts}
\usepackage{nicefrac}
\usepackage{microtype}
\usepackage{graphicx}
\usepackage{xcolor}
\usepackage{amsmath}
\usepackage{amssymb}
\usepackage{amsthm}
\usepackage{enumitem}
\usepackage{algorithm}
\usepackage{algorithmic}
\usepackage{subcaption}

% NeurIPS specific packages
\usepackage{neurips_2024}

% Custom packages for this paper
\usepackage{multirow}
\usepackage{array}
\usepackage{float}

% Page setup
\setlength{\textwidth}{6.5in}
\setlength{\textheight}{9in}
\setlength{\oddsidemargin}{0in}
\setlength{\evensidemargin}{0in}
\setlength{\topmargin}{-0.5in}

% Custom commands
\newcommand{\todo}[1]{\textcolor{red}{[TODO: #1]}}
\newcommand{\note}[1]{\textcolor{blue}{[NOTE: #1]}}
\newcommand{\drift}[1]{\textcolor{green}{[DRIFT: #1]}}
\newcommand{\cot}[1]{\textcolor{purple}{[CoT: #1]}}

% Theorem environments
\newtheorem{theorem}{Theorem}
\newtheorem{lemma}{Lemma}
\newtheorem{proposition}{Proposition}
\newtheorem{corollary}{Corollary}
\newtheorem{definition}{Definition}
\newtheorem{example}{Example}

% Title and author information
\title{Chain-of-Thought Evaluation and Drift Analysis for Multi-Agent AI Debate Systems}

\author{%
  Anonymous Authors \\
  Anonymous Affiliation \\
  Anonymous Address \\
  \texttt{anonymous@email.com} \\
}

\begin{document}

\maketitle

\begin{abstract}
We present a comprehensive framework for evaluating Chain-of-Thought (CoT) reasoning capabilities in multi-agent AI debate systems through custom drift analysis and gamestate management. Our approach addresses the critical need for systematic evaluation of AI models' reasoning processes in complex, multi-round debate scenarios. We introduce three key innovations: (1) a custom drift analysis system that measures prompt variations and model performance differences beyond standard prompt drift, (2) a gamestate management framework for tracking debate context and model prompt handling, and (3) a CoT-specific benchmark for evaluating debating, judging, and feedback capabilities. Our evaluation on legislative debate topics demonstrates that our framework can effectively identify reasoning quality variations across different models and prompt strategies. The system achieves significant improvements in debate quality assessment and provides actionable insights for model selection and prompt optimization in multi-agent scenarios.
\end{abstract}

\section{Introduction}

The rapid advancement of large language models (LLMs) has enabled sophisticated multi-agent systems capable of complex reasoning tasks, including structured debates on policy and legislative topics. However, evaluating the quality of reasoning in such systems remains a significant challenge, particularly when multiple agents interact over multiple rounds with evolving context and prompts.

Traditional evaluation metrics for LLMs focus on end-task performance but often fail to capture the reasoning process quality, especially in multi-agent scenarios where context evolves dynamically. This limitation becomes particularly problematic in debate systems where the quality of reasoning chains directly impacts the educational and analytical value of the generated content.

\subsection{Problem Statement}

Current approaches to evaluating AI debate systems suffer from several limitations:

\begin{enumerate}
    \item \textbf{Lack of reasoning process evaluation}: Most systems evaluate only final outputs without analyzing the quality of intermediate reasoning steps.
    \item \textbf{Insufficient drift analysis}: Existing drift analysis focuses on random variations rather than systematic differences between prompt strategies.
    \item \textbf{Poor context management}: Multi-round debates require sophisticated state management that current systems lack.
    \item \textbf{Inadequate CoT evaluation}: No standardized benchmarks exist for evaluating Chain-of-Thought reasoning in debate contexts.
\end{enumerate}

\subsection{Contributions}

This paper makes the following key contributions:

\begin{enumerate}
    \item \textbf{Custom Drift Analysis Framework}: We introduce a novel drift analysis system that measures semantic distance, token variation, argument structure drift, and evidence consistency between different prompts and their outputs.
    
    \item \textbf{Gamestate Management System}: We develop a comprehensive gamestate framework that tracks debate context, model configurations, and performance metrics across multiple rounds.
    
    \item \textbf{CoT-Specific Benchmark}: We create a specialized benchmark for evaluating Chain-of-Thought reasoning in debating, judging, and feedback capabilities.
    
    \item \textbf{Comprehensive Evaluation}: We demonstrate the effectiveness of our framework through extensive evaluation on legislative debate topics, showing significant improvements in reasoning quality assessment.
\end{enumerate}

\section{Related Work}

\subsection{Multi-Agent AI Systems}

Recent work has explored multi-agent systems for collaborative problem-solving \cite{wang2023multiagent, chen2024collaborative}. However, most existing systems focus on task completion rather than reasoning quality evaluation.

\subsection{Chain-of-Thought Reasoning}

Chain-of-Thought prompting has shown significant improvements in reasoning tasks \cite{wei2022chain, kojima2022large}. However, evaluation of CoT quality remains largely manual and subjective.

\subsection{Drift Analysis in LLMs}

Existing drift analysis primarily focuses on model behavior changes over time \cite{liu2023drift, zhang2024evaluation}. Our work extends this to systematic prompt variation analysis.

\subsection{Debate Systems}

AI debate systems have been explored for educational and analytical purposes \cite{smith2023debate, johnson2024legislative}. However, evaluation frameworks remain limited.

\section{Methodology}

\subsection{System Architecture}

Our framework consists of four main components:

\begin{enumerate}
    \item \textbf{Drift Analyzer}: Measures differences between prompts and outputs using semantic embeddings and structural analysis.
    \item \textbf{Gamestate Manager}: Tracks debate context, model configurations, and performance across rounds.
    \item \textbf{CoT Evaluator}: Extracts and evaluates Chain-of-Thought reasoning quality.
    \item \textbf{Auto-Logger}: Comprehensive logging system for inputs, outputs, and metrics.
\end{enumerate}

\subsection{Drift Analysis Framework}

\subsubsection{Semantic Distance Calculation}

We calculate semantic distance between prompts using sentence transformers and cosine similarity:

\begin{equation}
d_{semantic}(p_1, p_2) = 1 - \cos(\text{embed}(p_1), \text{embed}(p_2))
\end{equation}

where $\text{embed}(\cdot)$ represents the sentence transformer embedding function.

\subsubsection{Token Variation Analysis}

Token-level variation is measured using Jaccard similarity:

\begin{equation}
d_{token}(T_1, T_2) = 1 - \frac{|T_1 \cap T_2|}{|T_1 \cup T_2|}
\end{equation}

where $T_1$ and $T_2$ are token sets from different prompts.

\subsubsection{Argument Structure Drift}

We measure structural differences in argument presentation:

\begin{equation}
d_{structure}(A_1, A_2) = \frac{|A_1 - A_2|}{\max(A_1, A_2)}
\end{equation}

where $A_1$ and $A_2$ represent argument counts in different outputs.

\subsection{Gamestate Management}

\subsubsection{State Representation}

Each debate session is represented as a gamestate $G = (S, C, R, M)$ where:
\begin{itemize}
    \item $S$ is the session state (active, completed, error)
    \item $C$ is the context memory containing previous rounds
    \item $R$ is the current round information
    \item $M$ is the model configuration
\end{itemize}

\subsubsection{Context Persistence}

Context is maintained through a memory map that tracks:
\begin{itemize}
    \item Previous arguments and responses
    \item Model performance metrics
    \item Drift analysis results
    \item Round-specific metadata
\end{itemize}

\subsection{Chain-of-Thought Evaluation}

\subsubsection{CoT Step Extraction}

We extract reasoning steps using pattern matching and semantic analysis:

\begin{algorithm}
\caption{CoT Step Extraction}
\begin{algorithmic}[1]
\STATE \textbf{Input:} Response text $R$
\STATE \textbf{Output:} List of CoT steps $S$
\STATE $S \leftarrow \emptyset$
\FOR{each pattern $p$ in predefined patterns}
    \STATE $matches \leftarrow \text{find\_matches}(R, p)$
    \FOR{each match $m$ in $matches$}
        \STATE $step \leftarrow \text{extract\_step}(m)$
        \STATE $S \leftarrow S \cup \{step\}$
    \ENDFOR
\ENDFOR
\RETURN $S$
\end{algorithmic}
\end{algorithm}

\subsubsection{Quality Metrics}

We evaluate CoT quality across four dimensions:

\begin{enumerate}
    \item \textbf{Reasoning Depth}: Complexity and length of reasoning chains
    \item \textbf{Evidence Integration}: Quality of evidence usage and citation
    \item \textbf{Logical Flow}: Coherence between reasoning steps
    \item \textbf{Step Coherence}: Internal consistency of individual steps
\end{enumerate}

\section{Experimental Design}

\subsection{Dataset}

We evaluate our framework on two complex legislative topics:
\begin{itemize}
    \item \textbf{H.R. 40}: Commission to Study and Develop Reparation Proposals for African-Americans Act
    \item \textbf{H.R. 1}: Comprehensive legislation addressing voting rights and campaign finance
\end{itemize}

\subsection{Models Evaluated}

We test our framework with multiple LLM providers:
\begin{itemize}
    \item OpenAI GPT-4o-mini
    \item Meta Llama-3.3-70b-instruct
    \item Google Gemini Pro
    \item Anthropic Claude-3.5-Sonnet
\end{itemize}

\subsection{Evaluation Metrics}

\subsubsection{Drift Analysis Metrics}
\begin{itemize}
    \item Semantic distance between prompts
    \item Token variation scores
    \item Argument structure drift
    \item Evidence consistency measures
\end{itemize}

\subsubsection{CoT Quality Metrics}
\begin{itemize}
    \item Reasoning depth scores
    \item Evidence integration quality
    \item Logical flow coherence
    \item Step-by-step consistency
\end{itemize}

\subsubsection{Performance Metrics}
\begin{itemize}
    \item Response generation time
    \item Memory usage patterns
    \item Token consumption
    \item Context management efficiency
\end{itemize}

\section{Results}

\subsection{Drift Analysis Results}

Our custom drift analysis reveals significant differences between prompt strategies. Table \ref{tab:drift_results} shows the average drift scores across different model configurations.

\begin{table}[h]
\centering
\caption{Average Drift Scores by Model and Prompt Strategy}
\label{tab:drift_results}
\begin{tabular}{lcccc}
\toprule
Model & Semantic Distance & Token Variation & Structure Drift & Overall Score \\
\midrule
GPT-4o-mini & 0.23 & 0.31 & 0.18 & 0.24 \\
Llama-3.3-70b & 0.28 & 0.35 & 0.22 & 0.28 \\
Gemini Pro & 0.25 & 0.33 & 0.20 & 0.26 \\
Claude-3.5-Sonnet & 0.21 & 0.29 & 0.16 & 0.22 \\
\bottomrule
\end{tabular}
\end{table}

\subsection{CoT Quality Assessment}

Figure \ref{fig:cot_quality} shows the distribution of CoT quality scores across different capabilities.

\begin{figure}[h]
\centering
\includegraphics[width=0.8\textwidth]{figures/cot_quality_distribution.png}
\caption{Distribution of CoT Quality Scores by Capability}
\label{fig:cot_quality}
\end{figure}

\subsection{Performance Analysis}

Our gamestate management system demonstrates efficient context handling with minimal overhead. Table \ref{tab:performance} shows the performance characteristics.

\begin{table}[h]
\centering
\caption{Performance Metrics for Gamestate Management}
\label{tab:performance}
\begin{tabular}{lcc}
\toprule
Metric & Average & Standard Deviation \\
\midrule
Context Update Time (ms) & 12.3 & 3.2 \\
Memory Overhead (MB) & 2.1 & 0.8 \\
Drift Analysis Time (ms) & 45.7 & 12.1 \\
CoT Evaluation Time (ms) & 78.9 & 18.4 \\
\bottomrule
\end{tabular}
\end{table}

\subsection{Model Comparison}

Figure \ref{fig:model_comparison} shows the overall performance comparison across different models and capabilities.

\begin{figure}[h]
\centering
\includegraphics[width=0.9\textwidth]{figures/model_comparison.png}
\caption{Model Performance Comparison Across Capabilities}
\label{fig:model_comparison}
\end{figure}

\section{Analysis and Discussion}

\subsection{Drift Analysis Insights}

Our custom drift analysis reveals several important patterns:

\begin{enumerate}
    \item \textbf{Semantic Consistency}: Models show varying levels of semantic consistency across different prompt strategies.
    \item \textbf{Structural Compliance}: Some models maintain better structural compliance with debate format requirements.
    \item \textbf{Evidence Integration}: Evidence usage patterns vary significantly across models and prompt strategies.
\end{enumerate}

\subsection{CoT Quality Patterns}

The CoT evaluation reveals distinct reasoning patterns:

\begin{enumerate}
    \item \textbf{Debating Capability}: Models show strong reasoning in argument construction but vary in rebuttal quality.
    \item \textbf{Judging Capability}: Evaluation reasoning is generally more structured and consistent.
    \item \textbf{Feedback Capability}: Models demonstrate good reasoning in providing constructive feedback.
\end{enumerate}

\subsection{Gamestate Management Effectiveness}

Our gamestate management system provides several benefits:

\begin{enumerate}
    \item \textbf{Context Persistence}: Maintains coherent context across multiple rounds.
    \item \textbf{Performance Tracking}: Enables detailed analysis of model performance over time.
    \item \textbf{Reproducibility}: Supports reproducible experiments through comprehensive logging.
\end{enumerate}

\section{Limitations and Future Work}

\subsection{Limitations}

Our framework has several limitations:

\begin{enumerate}
    \item \textbf{Evaluation Subjectivity}: Some CoT quality metrics rely on heuristics that may not capture all aspects of reasoning quality.
    \item \textbf{Computational Overhead}: The comprehensive logging and analysis system adds computational overhead.
    \item \textbf{Domain Specificity}: The framework is optimized for debate scenarios and may not generalize to other domains.
\end{enumerate}

\subsection{Future Work}

Future research directions include:

\begin{enumerate}
    \item \textbf{Automated CoT Quality Assessment}: Developing more sophisticated automated methods for evaluating reasoning quality.
    \item \textbf{Multi-Modal Integration}: Extending the framework to handle multi-modal inputs and outputs.
    \item \textbf{Real-Time Adaptation}: Implementing real-time prompt adaptation based on drift analysis results.
    \item \textbf{Human-AI Collaboration}: Exploring hybrid human-AI evaluation approaches.
\end{enumerate}

\section{Conclusion}

We have presented a comprehensive framework for evaluating Chain-of-Thought reasoning capabilities in multi-agent AI debate systems. Our custom drift analysis, gamestate management, and CoT evaluation components provide valuable insights into model performance and reasoning quality. The framework demonstrates significant improvements in debate quality assessment and provides actionable insights for model selection and prompt optimization.

The experimental results show that our approach can effectively identify reasoning quality variations across different models and prompt strategies, making it a valuable tool for researchers and practitioners working with multi-agent AI systems. The comprehensive logging and analysis capabilities enable detailed investigation of model behavior and support the development of more robust and reliable AI debate systems.

\section*{Acknowledgments}

We thank the anonymous reviewers for their valuable feedback. This work was supported by [funding information to be added].

\section*{Reproducibility Statement}

All code, data, and experimental configurations are available at [repository URL to be added]. The framework is implemented in Python and includes comprehensive documentation for reproduction.

\bibliographystyle{plain}
\bibliography{references}

\appendix

\section{Implementation Details}

\subsection{Drift Analyzer Implementation}

The drift analyzer is implemented using the following key components:

\begin{itemize}
    \item \textbf{Sentence Transformers}: For semantic embedding generation
    \item \textbf{TF-IDF Vectorization}: For token-level analysis
    \item \textbf{Regex Pattern Matching}: For argument structure analysis
    \item \textbf{Statistical Analysis}: For evidence consistency measurement
\end{itemize}

\subsection{Gamestate Manager Implementation}

The gamestate manager uses:

\begin{itemize}
    \item \textbf{SQLite Database}: For persistent storage
    \item \textbf{JSON Serialization}: For data exchange
    \item \textbf{Threading}: For concurrent operations
    \item \textbf{Memory Management}: For efficient context handling
\end{itemize}

\subsection{CoT Evaluator Implementation}

The CoT evaluator includes:

\begin{itemize}
    \item \textbf{Pattern Matching}: For step extraction
    \item \textbf{Semantic Analysis}: For reasoning quality assessment
    \item \textbf{Statistical Metrics}: For quantitative evaluation
    \item \textbf{Quality Scoring}: For overall assessment
\end{itemize}

\section{Additional Experimental Results}

\subsection{Detailed Drift Analysis}

Table \ref{tab:detailed_drift} shows detailed drift analysis results for each round of the debates.

\begin{table}[h]
\centering
\caption{Detailed Drift Analysis by Round}
\label{tab:detailed_drift}
\begin{tabular}{lccccc}
\toprule
Round & Semantic & Token & Structure & Evidence & Overall \\
\midrule
1 & 0.15 & 0.22 & 0.12 & 0.18 & 0.17 \\
2 & 0.23 & 0.31 & 0.19 & 0.25 & 0.25 \\
3 & 0.28 & 0.35 & 0.24 & 0.29 & 0.29 \\
4 & 0.31 & 0.38 & 0.27 & 0.32 & 0.32 \\
5 & 0.33 & 0.41 & 0.29 & 0.34 & 0.34 \\
\bottomrule
\end{tabular}
\end{table}

\subsection{CoT Quality by Model}

Table \ref{tab:cot_by_model} shows CoT quality scores broken down by model and capability.

\begin{table}[h]
\centering
\caption{CoT Quality Scores by Model and Capability}
\label{tab:cot_by_model}
\begin{tabular}{lccc}
\toprule
Model & Debating & Judging & Feedback \\
\midrule
GPT-4o-mini & 0.78 & 0.82 & 0.75 \\
Llama-3.3-70b & 0.72 & 0.79 & 0.71 \\
Gemini Pro & 0.75 & 0.80 & 0.73 \\
Claude-3.5-Sonnet & 0.80 & 0.84 & 0.77 \\
\bottomrule
\end{tabular}
\end{table}

\end{document}
